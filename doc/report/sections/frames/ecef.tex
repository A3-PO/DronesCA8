\section{Earth-Centered Earth-Fixed (ECEF) Coordinate System}\label{sec:ecef}

\paragraph{}The Earth-Centered Earth-Fixed (ECEF) is a cartesian system which rotates with the Earth around its spin axis (blue lines in Figure \ref{fig:Geodetic1}), assigning a fixed set of coordinates for a specific point on the Earth's surface. The origin and axis of this system are defined as follows:
\begin{itemize}
\item{The origin (\textbf{$O_{e}$}) is located at the center of mass of the Earth. Hence the so-called s"Earth-Centered".}
\item{The Z-axis (\textbf{$Z_{e}$})} is set along the International Reference Pole (IRP).
\item{The X-axis (\textbf{$X_{e}$})} is set along the International Reference Meridian (IRM), intersecting the sphere (or elipsoid) of the Earth at 0$^{\circ}$ latitude (Equator) and 0$^{\circ}$ longitude (Greenwich).
\item{The Y-axis (\textbf{$Y_{e}$})} is orthogonal to both the Z and X axis described before by the usual right-hand rule.
\end{itemize}

\paragraph{}Even though ECEF coordinate frame is not commonly chosen as the final form to express the coordinates, it is frequently used as an intermediate step to convert between the \textbf{WGS84} geodetic system to the North-East-Down (NED) coordinate system as it will be shown later on.