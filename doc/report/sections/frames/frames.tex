\chapter{Frames}\label{ch:frames}

In navigation, guidance, and control of an aircraft or rotorcraft, there are several
coordinate systems (or frames) intensively used in design and analysis. For ease of references, we summarize in this chapter the coordinate systems
adopted in our work, which include:

\begin{itemize}
\item{The Geodetic Coordinate System}
\item{The Earth-Centered Earth-Fixed (ECEF) Coordinate System}
\item{The local North-East-Down (NED) Coordinate System}
\item{The vehicle-carried NED Coordinate System}
\item{The Body Coordinate System}
\end{itemize}

The relationships among these coordinate systems, i.e., the coordinate transforma-
tions, are also introduced.
We need to point out that miniature UAV rotorcraft are normally utilized at low
speeds in small regions, due to their inherent mechanical design and power limi-
tation. This is crucial to some simplifications made in the coordinate transforma-
tion (e.g., omitting unimportant items in the transformation between the local NED
frame and the body frame). For the same reason, partial transformation relationships
provided in this chapter are not suitable for describing flight situations on the oblate
rotating earth.