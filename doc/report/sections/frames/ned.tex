\section{North-East-Down (NED) Coordinate System}\label{sec:ned}

\paragraph{} The North-East-Down is a geographical coordinate system fixed to the Earth's surface, and, precisely because of this, is also known as \textbf{local tangent plane (LTP)} or ground coordinate system (green lines in Figure \ref{fig:Geodetic1}). The origin and axis of this system are defined as follows: 
\begin{itemize}
\item{The origin (\textbf{$O_{n}$}) is located at a arbitrary point on the Earth's surface.}
\item{The X-axis (\textbf{$X_{n}$}) points towards the geodetic north.}
\item{The Y-axis (\textbf{$Y_{n}$}) points towards the geodetic east.}
\item{The Z-axis (\textbf{$Z_{n}$}) points downward along the ellipsoidal normal, set by the right-hand rule.}
\end{itemize}

\paragraph{} The local NED frame plays a very important role in flight control and navigation.
Navigation of small-scale UAs is normally carried out within this frame. It will later be shown that during this work 2 different local NED frames would be described. One being with the origin fixed at the Ground Station location and one with an origin at the center of gravity of the UA at all times. This is the so-called \textbf{Vehicle-carried NED}, represented with the \textit{nv} subscript in Figure \ref{fig:NED1}.\\
Note that, strictly speaking, the axis of the Vehicle-carried NED are not completely aligned with the ones of the local NED , varying due to the movement of the vessel. However, for smalls aircrafts the directional difference is completely neglectable. Furthermore, $h = -z$ is often used to denote the actual height of the UA.

\paragraph{} Now, to define a point within this local frame, spherical coordinates will be used. In mathematics, a spherical coordinate system is a coordinate system for three-dimensional space where the position of a point is specified by three numbers: the \textbf{radial distance} of that point from a fixed origin ($\rho$), its \textbf{polar angle} measured from a fixed zenith direction ($\theta$), and the \textbf{elevation angle} of its orthogonal projection on a reference plane, measured from a fixed reference direction on that plane ($\phi$).\\
The way these spherical coordinates are defined during this specific project are shown in Figure \ref{fig:Spherical1}.
\begin{figure}[H]
   \centering
    \includestandalone[width=.60\textwidth]{figures/3D_Sphcoord} 
    \caption{Spherical Coordinates in NED Frame.}
    \label{fig:Spherical1}
\end{figure}

\paragraph{} Note how this cartesian coordinate system is defined such that $\theta = 0$ is aligned with the east ($y$) axis of the NED frame and  $\phi > 0$ is defined for negative values or $z$, i.e., positive height on the surface of the Earth.
Additionally, it is necessary to define a unique set of spherical coordinates for each point, and therefore restricting the range of these parameters is required. In our case the coordinates will be limited as follows:
\begin{align*}
& \rho \geq 0 \\
& -\pi \leq \theta < \pi \\
& \frac{-\pi}{2} \leq \phi \leq \frac{\pi}{2}
\label{eq:los_distToHorizon}
\end{align*}
These constraints define the conversion between Cartesian and Spherical coordinates such that:
\begin{align*}
x &=  \rho\cos\phi\sin\theta  & \rho &= \sqrt{x^{2} + y^{2} + z^{2}} \\
y &= \rho\cos\phi\cos\theta   & \theta &= \arctan\left(x\right)\\
z &= \rho\sin\phi       & \phi &=  \arctan\left(-z\right)
\label{eq:los_distToHorizon}
\end{align*} 

