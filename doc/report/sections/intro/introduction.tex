\chapter{Introduction}\label{ch:intro}

The current legislation prevents the use of UAVs out of the line of sight meaning that the users have to see the aircraft when they are piloting it. However, this legislation will be changed, so it will be possible to control these devices without seeing them. A way to keep a connection with the aircraft is to improve the communication systems with the ground station. Hence, this regulation will allow the introduction of UAVs in new fields where they can optimize humans' tasks. For instance, these new UAVs can be used to patrol huge forest areas in order to detect fire starts (aircraft equipped with a thermal camera), instead of having some fire lookout towers preventing this catastrophe. This surveillance, that has to be done day and night in large areas, would not involve risks for human beings and would also be cheaper. 

%This solution that would improve the actual one that is lookout towers and human patrols...%

%PREVIOUS FIRST PARAGRAPH% 
%The current legislation prevents the people to control UAVs out of the line of sight, which means that the users need to see the aircraft when they are piloting it. However, this legislation will change and it will be possible to control this device using a camera and a system which allows the communication between the ground station and the UAV. So, in order to have these mechanisms, it is necessary to solve some issues that, until now, were not studied. The connection between them is one of the main problems because it requires new skills that the old devices don't have. This new legislation allows the introduction of UVAs in new fields where they can perform some functions that the humans are not able to do. For instance, these new UAVs can be used to patrol huge forest areas in order to detect fire starts, instead of having some fire lookout towers that are ineffective against this catastrophe. This surveillance, that has to be done day and night in large areas, would be cheaper without risks for human beings.%

%%%%Currently, the

In this project, the challenge is to maintain the connection between the UAV and the ground station, not only when the distance is a few meters, but also when it is tens of kilometers. Such a zone can include mountainous areas, which are physical barriers, flat areas and zones where the atmosferic conditions can be a problem for the navigation (for example, windy or rainy places). Due to the difficulty or, sometimes, the impossibility of exceed these obstacles, it is mandatory to improve the connection to decrease their effect. In other words, the aircraft needs to be prepared for this barriers in order to have a reliable communication with its ground station. 

Thus, the main goal of this project is to increase the distance between the devices using two tracking antennas (one on the ground and one on the aircraft) that are able to keep the communication. Directional antennas can solve this problem  because it is possible to use their main lobes to guarantee a reliable connection. As a result, it will be necessary to develop a system capable of controlling both antennas to make them point at each other.

%PREVIOUS THIRD PARAGRAPH%
%In this project, the challenge is to maintain the connection between the UAV and the ground station when the distance that separates them is some tens of kimometers. This huge distance involves not only flat areas, but also mountainous areas which are physical barriers that don’t allow a correct communication and where the atmospheric conditions can be really hard (for example, windy or rainy places). However, the aircraft needs to be prepared structurally for a set of conditions in which the communication between the ground station and the UAV is reliable. The signal emitted and received by both antennas needs to be strong enough to overtake every obstacle. Thus, the main goal of this project is to communicate with a UAV through two tracking antennas (one on the ground and one on the other on the aircraft) when the distance between them is too big. In order to improve the communication it will be necessary to explore the best qualities of the antennas. In other words, directional antennas will be the best choice because it is possible to use their main lobes to guarantee a reliable connection. So that, it will be necessary to develop a system able to control both antennas to put them pointing at each other.%


Before the new legislation, due to the fact that the UAV was flying close to the ground station, the signal was strong enough to maintain the contact. For example, omnidirectional antennas are able to do this task although they don't have any prime direction. On the other hand, when the distance between the aircraft and the ground station increases, the contact becomes difficult because the loss of signal increases a lot. The limited frequency and bandwidth by means of law and the limitated antenna power and size are other constraints that can compromise the detection of the signal. 


%Our case: limited frequency and bandwidth by means of the Law. the can't emit a signal.......(read more about this)
%Limited antenna power and size. The power of the antennas is not infinite so it is difficult to make sure that the signal will be received by the receiver. The mountains are physical barriers because the signal can't pass through it - the drone needs to fly in a high altitude
%The weather degradates the signal and it will be difficult to communicate through certain conditions. The only solution is to try to point the antennas to each other because it will increase the probability of communication.
%Controller used controls 2 angles of both antennas (just say that not explain anything).
