\chapter{Modelling}\label{ch:model}

Due to the fact that it is essential to know how the systems work before applying them in practical situations, the replication of real-life situations is helpful to understand their behaviour. In order to be able to recreate the real-life events, it is indispensable to describe the systems regarding their characteristics. The characterization of these systems is made byusing models.  

A model is a simplified mathematical representation of a system at some specific point in time or space intended to promote understanding of the real system. Models allow us to reason about a system and make predictions about how a system will behave (simulation). Whether a model is a good model or not depends on the extent to which it promotes understanding. Since all models are simplifications of reality there is always a trade-off as to what level of detail is included in the model. If too little detail is included in the model one runs the risk of missing relevant interactions and the resultant model does not promote understanding. If too much detail is included, the model may become overly complicated and actually preclude the development of understanding. 

The model of the moving angle system, the optimal angle and the antenna will be described in the next sections of this chapter.