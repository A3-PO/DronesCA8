\chapter{Results and Discussion}\label{ch:results}

This section presents four different scenarios where the movements of the antennas in the UA and in the GS are described, taking into account the information presented in the 3D simulation (section \ref{sec:3d_sim}).

These four scenarios are the Angle Range, the Curvature of the Earth, Above the GS and the Mountain. Each scenario was created in order to cover a possible real-life situation and, thus, analyse the viability of the developed tracking method.

The main goal of the \textbf{Angle Range} scenario is mostly to study the behaviour of the azimuth angle and to describe the performance of the different controllers. The \textbf{Curvature of the Earth} shows essentially the variation of the elevation angle. The principal objective of the scenario \textbf{Above the GS} is to analyse how fast the antennas can be when the UA overflies the GS. Finally, the \textbf{Mountain} demonstrates how the system reacts when the GS loses the connection with the UA. 