\section{Scenario 4}\label{sec:scenario4}
The purpose of this scenario is to show what can happen when the UAS loses line-of-sight because of a mountain, as seen on the map in Figure \ref{fig:s4_map}. Also, the controller used in the simulation is a PD controller.

\begin{figure}[H]
\hfill
\subfigure[UAS Map Positioning]{\includegraphics[scale=0.33]{figures/scenario_4_map.png}}
\hfill
\subfigure[LOS and Distance]{\includegraphics[scale=0.33]{figures/scenario_4_los.png}}
\hfill
\caption{Mountain Scenario}
\label{fig:s4_map}
\end{figure}

\subsection{UA}
In Figure \ref{fig:s4_ua} the angle tracking of the UA antenna can be seen.

\begin{figure}[H]
\centering
\includegraphics[scale=0.75]{figures/scenario_4_ua.png}
\caption{Azimuth and elevation angles of UA following the optimal angle}
\label{fig:s4_ua}
\end{figure}

\subsection{GS}
In Figure \ref{fig:s4_gs} the angle tracking of the GS antenna can be seen.

\begin{figure}[H]
\centering
\includegraphics[scale=0.75]{figures/scenario_4_gs.png}
\caption{Azimuth and elevation angles of GS following the optimal angle}
\label{fig:s4_gs}
\end{figure}

\subsection{Power}
In Figure \ref{fig:s4_power} the power at the receiver antenna of the GS antenna can be seen.

\begin{figure}[H]
\centering
\includegraphics[scale=0.75]{figures/scenario_4_power.png}
\caption{Power at the receiver's antenna (GS)}
\label{fig:s4_power}
\end{figure}