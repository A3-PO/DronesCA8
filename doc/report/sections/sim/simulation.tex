\chapter{Simulation}\label{ch:sim}

\paragraph{} Inside this section a series of simulations will be performed in order to determine, first, what the limits of this specific scenario are and, secondly, how the actual system would turn out to be.
The general approach carried out during this project leaves up a lot parameters to decide such as the antennas used (both for ground station and UAV) or the noise accepted in the system. This choices will be the ones limiting the maximum distance possible for the system to still work.

\paragraph{} Later on, a 2D simulation followed up by the 3D extension after will be derived. Note that a 2-dimension simulation implies a 1 degree of freedom (one angle) for the controller, whereas the augmented 3-dimension version includes both azimuth and elevation angle to control.
\section{Parameter simulation}
\begin{itemize}
\item{Viable distances}
\item{Antenna Gains}
\item{Noises}
\item{Power receiver thresholds...}
\end{itemize}
