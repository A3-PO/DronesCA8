\chapter{Telecommunication}\label{ch:tele}

Drone communication links are generally either radio frequency (RF) or lasercom (optical). Even though both types currently suffer from bandwidth limitations, lasercom could surpass RF in terms of airborne data transfer rate. However, RF will continue to dominate at the lower altitudes for some time into the future because of its better all-weather capability. Additionally, RF links have the advantage of being much more efficient and usually much less complex than lasercom links.
Data rates for RF links have traditionally been restricted due to limited spectrum and
minimization of communication system size, weight, and power.

Radio waves are a form of electromagnetic radiation with frequencies ranging from 3 kHz to 300 GHz. The size and profile constraints of small drones cannot facilitate communication links on the low-frequency (large wavelength) end of the spectrum. 
On the other hand, atmospheric attenuation as well as attenuation by rain, both
exponential in character, becomes a serious limiting factor to the link distance in
the millimeter wavelength region (>300MHz). 
	
Due to these limitations, small drone communication links are restricted to the very high frequency (VHF), the ultrahigh frequency (UHF), and rarely the super-high frequency (SHF) bands.

