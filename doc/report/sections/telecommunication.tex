\chapter{Telecommunication}\label{ch:telecommunication}

Drone communication links are generally either radio frequency (RF) or lasercom (optical). Even though both types currently suffer from bandwidth limitations, lasercom could surpass RF in terms of airborne data transfer rate. However, RF will continue to dominate at the lower altitudes for some time into the future because of its better all-weather capability. Additionally, RF links have the advantage of being much more efficient and usually much less complex than lasercom links.
Data rates for RF links have traditionally been restricted due to limited spectrum and
minimization of communication system size, weight, and power.

Radio waves are a form of electromagnetic radiation with frequencies ranging from 3 kHz to 300 GHz. The size and profile constraints of small drones cannot facilitate communication links on the low-frequency (large wavelength) end of the spectrum. 
On the other hand, atmospheric attenuation as well as attenuation by rain, both
exponential in character, becomes a serious limiting factor to the link distance in
the millimeter wavelength region (>300MHz). 
	
Due to these limitations, small drone communication links are restricted to the very high frequency (VHF), the ultrahigh frequency (UHF), and rarely the super-high frequency (SHF) bands.

\section{Telemetry}
One of the main goal of drone surveillance is to procure relevant data. This is done by means of telemetry which is an automated communication process. In this process the data collected is transmitted to a receiving equipment for further processing and monitoring. 

!!! ADD MORE INFO !!!

!!! ADD FIGURE !!!

!!! HOW WE USE IT ? !!!

\section{MAVLink protocol}
When speaking of digital communication it is required to have a certain communication protocol. Given the fact that, this project deals with drones, a special protocol, namely Micro Air Vehicle Link (MAVLink) has been taken into consideration. This protocol is used in a GNS and drones scenario, where inter-communication between systems is needed in order to transmit GPS location, heading angle and speed.  

\subsection{Packet structure}
For further understanding the packet structure of such a protocol is needed as seen in Table \ref{tab:mavlink}.

\begin{table}[h]
	\centering
	\begin{tabular}{|c||c|c|}
		\hline
		Field name       & Index (Bytes)  & Purpose											     \\ \hline\hline
		Start-of-frame   &      0         & Start of frame transmission 							   \\ \hline
		Pay-load-length  &      1         & Length of payload (n)       							   \\ \hline
		Packet sequence  &      2    	  & Sent sequence counter (detect packet loss)                 \\ \hline
		System ID        & 		3		  & Sending system identification 							   \\ \hline
		Component ID     & 		4 		  & Sending component identification 						   \\ \hline
		Message ID       & 		5 		  & Message identification (correctly decoded)      		   \\ \hline
		Payload          &   6 to (n+6)   & Data into the message, depends on Message ID        	   \\ \hline
		CRC              & (n+7) to (n+8) & Check-sum of packet (excluding packet start sign)          \\ \hline
	\end{tabular}
	\caption{MAVLink packet structure}
	\label{tab:mavlink}
\end{table}

The CRC field ensures message integrity of each packet. Another function of the CRC is to establish that both sender and receiver agree on the message transfer. Additionally, a seed value is appended at the end of the data when computing the CRC, generated with every new message.

\subsection{Messages}
As stated above the payload from the packets are MAVLink messages. Also, every message can be identified by its ID field on the packet. Additionally, an XML document (in MAVLink source) has the definition of all the data stored on that payload. A sample of such an XML document can be seen in Figure \ref{fig:mav_msg} which describes a message with an ID 24 giving GPS relevant information.

\begin{figure}[h]
	\centering
	\includegraphics[scale=0.5]{figures/mavlink_msg.jpg}
	\caption{MAVLink message XML document}
	\label{fig:mav_msg}
\end{figure}

To be noted that the XML document describes the logical ordering of the fields for the protocol, not the actual wire format.


\section{Compute Link Budget}

\subsection{Line-Of-Sight Propagation}\label{subsec:los_propagation}
At low frequency (below approximately 3 MHz) radio signals travel as ground waves, which follow the Earth's curvature due to diffraction with the layers of the atmosphere.
However, at higher frequencies and in lower levels of the atmosphere, neither of these effects are significant. Thus any obstruction between the transmitting antenna (transmitter) and the receiving antenna (receiver) will block the signal, just like the light that the eye may sense. Therefore, since the ability to visually see a transmitting antenna (disregarding the limitations of the eye's resolution) roughly corresponds to the ability to receive a radio signal from it, the propagation characteristic of VHF and higher radio frequency (>30 MHz) paths is called line-of-sight. The farthest possible point of propagation is referred to as the radio horizon.

The radio horizon is the locus of points at which direct rays from an antenna are tangential to the surface of the Earth. If the Earth were a perfect sphere and there were no atmosphere, the radio horizon would be a circle.
This way the greatest distance at which a receiver can see the transmitter is explained in the following paragraph.

First, we are going to derive a general expression and after that apply it to a scenario with a drone and a basestation. In figure  \ref{fig:GeometricDist_general} the relationship between the height of the observer above sea level (O point) and the distance d which is between it and the horizon (H point) is shown. Finding this distance is done by the use of the pythagorean theorem. With some simple mathematical calculations the distance d is derived in the following:

\begin{equation}\label{eq:los_distToHorizon}
	(R+h)^2 = R^2+d^2\nonumber \\
	\Rightarrow R^2+2hR+h^2 = R^2+d^2 \Rightarrow d^2 = 2hR + h^2 \\
	\Rightarrow d = \sqrt{2hR + h^2}
\end{equation} 

\begin{figure}
    \hfill
    \subfigure[Geometrical distance to the horizon]{
    	\includegraphics[scale=4]{figures/GeometricDistanceToHorizonOneTriangle.png} 
		\label{fig:GeometricDist_general}}
	\hfill
    \subfigure[Geometrical distance from drone to GNS]{
    	\includegraphics[scale=0.3]{figures/GeometricDistanceToHorizonTwoTriangle.png} 
    	\label{fig:GeometricDist_droneBasestation}}
    \hfill
    \caption{Geometrical distance to the horizon, Pythahorean theorem}
\end{figure}

On figure \ref{fig:GeometricDist_droneBasestation} it is shown that the two objects are a drone and a basestation. Both of them wont be higher then approx 100 meter and since R is radius of the Earth, $2hR$ >> $h^2$ and $h^2$ is therefore neglected in equation \ref{eq:los_distToHorizon}. The two distances $D_D$ and $D_B$ have the same expressions in both cases:
\begin{align*}
	D_D [km] &= \sqrt{2\cdot R \cdot h_D + h_{D}^2} \approx \sqrt{2\cdot 6.378\cdot h_D} = \sqrt{12.756\cdot h_D} = 3.57\cdot \sqrt{h_D} \\
	D_B [km] &= \sqrt{2\cdot R \cdot h_B + h_{B}^2} \approx \sqrt{2\cdot 6.378\cdot h_B} = \sqrt{12.756\cdot h_B} = 3.57\cdot \sqrt{h_B}
\end{align*}

To calculate the distance $D_{DB}$:
\begin{align}
	D_{DB}[km]	 &= D_D + D_B \approx 3.57\cdot \sqrt{h_D} + 3.57\cdot \sqrt{h_B} = {3.57\cdot (\sqrt{h_D} + \sqrt{h_B}} )
\end{align}

\subsection{Example with drone = 100m and basestation = 20m}
Lets take an example if the drone is at $h_D = 100m$ and the basestation at $h_B = 20m$. The distance between the drone and the basestation is as follows:
\begin{equation*}
	D_{DB}[km] = 3.57\cdot (\sqrt{100} + \sqrt{20}) = 51.67km
\end{equation*}

\subsection{Free space path loss}\label{subsec:path_loss}
\paragraph{}
The free space path loss (FSPL) is the loss in signal strength that occurs when an electromagnetic wave travels over a line of sight path in free space. In these circumstances there are no obstacles that might cause the signal to be reflected, refracted, or that might cause additional attenuation. Equation \ref{eq:path_losses} represents the loss in signal strength in dB.

\begin{equation}\label{eq:path_losses}
	L_{FS} = 20\lg\left (\frac{4\pi \cdot d}{\lambda} \right)
\end{equation}

The wave length can also be described by a relationship between the frequency and the velocity of light. This relationship is described by the equation \ref{eq:vel_freq_wavelen1}.

\begin{equation}\label{eq:vel_freq_wavelen1}
	\lambda = \frac{c}{f}
\end{equation}

Explanation of the parameters.
\begin{itemize}
	\item d - distance from transmitter to receiver [m]
	\item $\lambda$ - wavelength of the signal [m]
	\item c - speed of light constant $3\cdot 10^8$ [m/s] 
	\item f - frequency of signal [Hz]
\end{itemize}

Considering our problem we will look into the worst case scenario, which would be maximum distance between the GNS and the drone. 
\begin{equation*}
	Assume 
	\begin{cases}
	d_{max} = \sqrt{x^2+y^2}\\
	\text{f} = f\text{GHz (Should be permitted by law})\\
	\end{cases}
\end{equation*}

Computing signal wavelength:
\begin{equation}\label{eq:vel_freq_wavelen2}
	\lambda = \frac{c}{f} 
	        = \frac{3\cdot 10^{8}}{f\cdot 10^{9}}
	        = \frac{3}{10f}m
\end{equation}

Computing path loss:
\begin{align*}\label{eq:path_loses_calc}
	L = 20\lg\left (\frac{4\pi d}{\lambda} \right) dB 
	 &= 20\lg\left (\frac{4\pi \sqrt{x^2+y^2}}{\frac{3}{10f}} \right) dB\\ 
	 &= 20\lg\left (\frac{4\pi \sqrt{x^2+y^2}\cdot 10f}{ 3} \right) dB
\end{align*}
\noindent \textbf{As an observation, higher distance value between GNS and drone will result in higher path loss.}

\subsection{Link Budget}\label{subsec:link_budget}
\paragraph{}
A link budget is accounting of all of the gains and losses from the transmitter, through the medium  to the receiver in a telecommunication system. It accounts for the attenuation of the transmitted signal due to propagation, as well as the antenna gains, feedline and miscellaneous losses. 
\begin{equation*}\label{eq:link_budget} 
 		\text{Received Power (dBm)} = \text{Transmitted Power (dBm)} + \text{Gains (dB)} - \text{Losses(dB)}
\end{equation*}

In more detailed a common radio link looks like this:

\begin{equation*}\label{eq:link_budget} 
 		P_{RX} = P_{TX} + G_{TX} - L_{TX} - L_{FS} - L_{M} + G_{RX} - L_{RX}
\end{equation*}

Note that decibels are logarithmic measurements, so adding decibels is equivalent to multiplying the actual numeric ratios.


\section{Fresnel zones}
Taking into account that the application at hand involves radio communication it is important to talk about the Fresnel zones. Thus, it can be seen in Figure \ref{fig:fresnel_zones} the three Fresnel zones on the transmission path between A and B. 

\begin{figure}[h]
	\centering
	\includegraphics[scale=0.65]{figures/fresnel_zones.png}
	\caption{Fresnel zones between transmitter and receiver}
	\label{fig:fresnel_zones}
\end{figure}


ADD MORE RELEVEANT THINGS 

! ! ! ! \url{https://en.wikipedia.org/wiki/Fresnel_zone}  ! ! ! !

