\pdfbookmark[0]{English title page}{label:titlepage_en}
\aautitlepage{%
  \englishprojectinfo{
    Double Tracking Antennas for \\
    UAS Communication %title
  }{%
    Multivariable Control %theme
  }{%
    Spring Semester 2016 %project period
  }{%
    CA832 % project group
  }{%
    %list of group members
    Alvaro Perez Ortega\\ 
    Kenny Lund Lafon\\
    Kelvin Kjærvik Pagels\\
    Robert-Octavian Popescu\\
    Orlando Bastos Vaz
  }{%
    %list of supervisors
    Anders La Cour Harbo
  }{%
    1 % number of printed copies
  }{%
    \today % date of completion
  }%
}{%department and address
  \textbf{Electronics and IT}\\
  Aalborg University\\
  \href{http://www.aau.dk}{http://www.aau.dk}
}{% the abstract
  In an Unamanned Aircraft System (UAS) scenario, one of the main goals is to secure Line-of-Sight (LOS) between the Unmanned Aicraft (UA) and the Ground Station (GS). Moreover, in most cases permanent communication between systems must be assured. Such that, tracking directional antennas in both ends have been considered to have assure this communication. 

  Furthermore, the tracking is done with the help of a DC servomotor which will turn each antenna. Additionally, different controller's have been tested and tunned for this specific application to achieve valuable results. Also, 2D and 3D simulations of the whole system have been made with the servomotor model and controller integrated.  
  
  The scope of the tracking is to achieve satisfactory link budget over long distances. In such a way, large areas can be covered in any application that requires permanent communication between GS and UA. 
}

\cleardoublepage
% {\selectlanguage{danish}
% \pdfbookmark[0]{Danish title page}{label:titlepage_da}
% \aautitlepage{%
%   \danishprojectinfo{
%     Double tracking antennas for drone communication %title
%   }{%
%     Multivariable control %theme
%   }{%
%     Spring 2016 %project period
%   }{%
%     Group: 832 % project group
%   }{%
%     %list of group members
%     Alvaro Perez Ortega\\ 
%     Kenny Lund Lafon\\
%     Kelvin Kjærvik Pagels\\
%     Robert-Octavian Popescu\\
%     Orlando Vaz
%   }{%
%     %list of supervisors
%     Anders La Cour Harbo
%   }{%
%     1 % number of printed copies
%   }{%
%     \today % date of completion
%   }%
% }{%department and address
%   \textbf{Elektronik og IT}\\
%   Aalborg Universitet\\
%   \href{http://www.aau.dk}{http://www.aau.dk}
% }{% the abstract
%   Her er resuméet
% }}