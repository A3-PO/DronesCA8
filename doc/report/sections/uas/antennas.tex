\section{Antennas}\label{sec:antennas}

As it was mentioned before the often approach to this kind of communication is to use directional antennas.
The reason to do this, is that a directional antennas are known for having a very high gain in a narrow beam width, whereas on the other hand, an omnidirectional antenna has a lower gain while having a very broad beam width.
Therefore, the use of this first type of antennas will allow us to improve the quality of the radio link as long as the antennas are pointing at each other.

The antennas used for the GS and the UA are different by means of weight and size. The preference of using them is based on the application at hand, thus for:
\begin{itemize}
	\item GS - Parabolic (grid) directional antenna 
	\item UA - Patch directional antenna
\end{itemize}

A parabolic antenna in an antenna that uses a parabolic reflector and a curved surface to direct the radio waves and its main advantage is that it has a high directivity. This type of antennas are able to produce a narrow beam width which allow them to have some of the highest gains Parabolic antennas, due to their high gain, are intensively used for carry telephone and television signals between nearby cities.

Patch antenna, which is the original type of microstrip, is a low profile antenna that can be mounted on a flat surface and it consists in a rectangular sheet of metal. These antennas are very useful because they are very thin and they provide a quite high directivity.

Later on, in Chapter \ref{ch:model} a modelling process to simulate these types of directional antennas will be addressed.