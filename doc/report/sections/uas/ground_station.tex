\section{Ground Station}\label{sec:gs}

The ground station is, generally, a terrestrial radio station designed for planetary and extra-planetary telecommunication with aircrafts and spacecrafts, respectively. For planetary telecommunications, the ground stations are located on the surface of the Earth and they communicate with the aircrafts by sending and receiving radio waves in a specific band of the spectrum. When a ground station successfully transmits radio waves to an aircraft or vice versa, it establishes a telecommunication link.

In the case of this project, the ground station is prepared to control the UA using an antenna and a computer to communicate, analyse the received data and send commands to the aircraft. As was described in the Section \ref{sec:drone}, in order to accomplish the desired task, the eBee was taken as an example. The monitorization of the movement of this aircraft is done using the eMotion software that is installed in the base station computer \cite{eBee}.

eMotion is a software that enables the user to plan, simulate, monitor and control the movement of the UA. In the planning part it is possible to import the background map and to draw a polygon that will correspond to the area that the user wants to cover. Then the user needs to define some parameters, like the ground resolution, and the eMotion will automatically generate a full flight plan, calculating the eBee’s required altitude and displaying its projected trajectory. It is also possible to correct some parameters of the generated trajectory that were not taking into account in the previous case, allowing, for instance, a fly over an uneven terrain \cite{eBee}.

Before the flight, it is possible to simulate the trajectory of the UA using eMotion's simulator. The simulation helps the user to understand how the camera will work during the flight and how will be the movement of the UA, without putting the aircraft at risk. After the simulation, while the vehicle is flying, eMotion provides a real-time information displaying all the flight parameters, the battery level and the image acquisition progress. During the flight, it is possible to manually control the UA using a tool that is available in this software. Every mistake can be corrected allowing the reconfiguration of the vehicle's flight plan and the landing point. Hence, this system allows a versatile control of the aircraft from the plan of the trajectory until the control during the flight \cite{eBee}.