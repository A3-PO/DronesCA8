\chapter{Unmanned Aerial System Overview}\label{ch:uas}

The main goal of this project is to try to increase the distance between the aircraft and the ground station using two tracking antennas. However, the increase of the distance will reduce the quality of the signal that is received and sent by both antennas. In order to keep the communication, it is essential to have the proper physical infrastuctures to be able to detect both strong and weak signals. These infrastuctures constitute the hardware and they cover every element connected with the vehicle, in the air, and the ground station, on the ground. The previous description of the system as a whole constitutes the Unmanned Aerial System, UAS.

The UAS illustrated in Figure \ref{fig:uas} can be decomposed in the following elements:
\begin{itemize}
	\item Unmanned Aerial Vehicle, UAV
	\item Ground Station, GS
	\item Antennas
	\item Survey camera
\end{itemize}

\begin{figure}[H]
	\centering
	\includegraphics[scale=0.33]{figures/uas.png}
	\caption{Unmanned Aerial System}
	\label{fig:uas}
\end{figure}

%In order to have some bounds of the system we can have a look at the drone performances:
%\begin{itemize}
%	\item 50 km/h - stability at high speeds 
%	\item 40 minutes - battery autonomy (30 minutes with payload)
%	\item 2400 m - maximum flight altitude (higher if take off from mountain site)
%	\item 50 km - maximum radio communication (with directional antennas)
%	\item under 1 m - absolute positioning of X-Y GPS
%	\item GPS return-to Home (automatically activates when radio link is lost)
%\end{itemize}


