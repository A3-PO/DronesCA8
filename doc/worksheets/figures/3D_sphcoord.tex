\documentclass[crop,tikz]{standalone} 

\begin{document}

%Angle Definitions
%-----------------
%set the plot display orientation
%synatax: \tdplotsetdisplay{\phi_d}{\theta_d}
\tdplotsetmaincoords{60}{110}

%define polar coordinates for some vector
%TODO: look into using 3d spherical coordinate system
\pgfmathsetmacro{\rvec}{.8}
\pgfmathsetmacro{\phivec}{30}
\pgfmathsetmacro{\thetavec}{60}
%
\begin{tikzpicture}[scale=5,tdplot_main_coords]

%set up some coordinates 
%-----------------------
    \coordinate (O) at (0,0,0);

%determine a coordinate (P) using (r,\phi,\theta) coordinates.  This command
%also determines (Pxy), (Pxz), and (Pyz): the xy-, xz-, and yz-projections
%of the point (P).
%syntax: \tdplotsetcoord{Coordinate name without parentheses}{r}{\phi}{\theta}
    \tdplotsetcoord{P}{\rvec}{\phivec}{\thetavec}

%draw figure contents
%--------------------
%draw the main coordinate system axes    
    \draw[thick,->] (0,0,0) -- (1,0,0) node[anchor=north east]{$x$};
    \draw[thick,->] (0,0,0) -- (0,1,0) node[anchor=north west]{$y$};
    \draw[thick,->] (0,0,0) -- (0,0,1) node[anchor=south]{$z$};


%draw a vector from origin to point (P) 
    \draw[-stealth,color=red] (O) -- (P) node[midway, above left] {\small{$\rho$}}node[above right] {$P$};

%draw projection on xy plane, and a connecting line   
    \draw[dashed, color=red] (O) -- (Pxy);
    \draw[dashed, color=red] (P) -- (Pxy);

%draw the angle \theta, and label it
%syntax: \tdplotdrawarc[coordinate frame, draw options]{center point}{r}{starting angle}{ending angle}{label options}{label}
    \tdplotdrawarc{(O)}{0.2}{0}{\thetavec}{anchor=north}{$\theta$}

%set the rotated coordinate system 
%syntax: \tdplotsetphiplanecoords{rotation-z-axis}{rotation-y-axis}{rotation-x-axis}
    \tdplotsetthetaplanecoords{\thetavec}

%draw phi arc and label, using rotated coordinate system
    \tdplotdrawarc[tdplot_rotated_coords]{(0,0,0)}{0.3}{\phivec}{90}{anchor=south west}{$\phi$}


\end{tikzpicture}
\end{document}