\chapter*{Technical Scenario}\label{ch:technical_scenario}

\section*{Problem parameters}
As mentioned before we need to assure the maximum distance of communication possible between basestation and the drone at a certain working frequency:

\begin{equation*}\label{eq:tech_parameters1} 
 	\begin{cases}
 		d_{max} = 50 km	\\
 		f = 2.4 GHz
 	\end{cases}
\end{equation*}

\section*{Antenna Overview}
The antennas used for the basestation and the UAV are different by means of weight and size. The preference of using them is based on the application at hand, thus for:
\begin{itemize}
	\item Basestation - Parabolic (grid) directional antenna 
	\item Drone - Patch directional antenna
\end{itemize}

A parabolic antenna in an antenna that uses a parabolic reflector and a curved surface to direct the radio waves and its main advantage is that it has a high directivity. This type of antennas are able to produce the narrowest beam widths which allow them to have some of the highest gains.

Parabolic antennas, due to their high gain, are intensively used for carry telephone and television signals between nearby cities. In our specific case, it’s possible to use this property to receive the information provided by the thermal camera through the UAS.

\begin{table}[h!]
\centering
	\begin{tabular}{|p{3cm}||p{3cm}|p{3cm}|}
		\hline
		Parameter & Basestation & Drone\\ \hline
		Type & Parabolic & Patch\\ \hline
		Polarization & Linear & Linear\\ \hline
		Gain & $24dB$ & $14dB$\\ \hline
		HPBW/$H(^{\circ})$ & $14^{\circ}$ & $45^{\circ}$\\ \hline
		HPBW/$V(^{\circ})$ & $10^{\circ}$ & $45^{\circ}$\\ \hline
	\end{tabular}
\end{table}