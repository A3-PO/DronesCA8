
\subsection*{Free space path loss}\label{subsection:path_loss}

The free space path loss (FSPL) is the loss in signal strength that occurs when an electromagnetic wave travels over a line of sight path in free space. In these circumstances there are no obstacles that might cause the signal to be reflected, refracted, or that might cause additional attenuation. Equation \ref{eq:path_losses} represents the loss in signal strength in dB.

\begin{equation}\label{eq:path_losses}
L_{FS} = 20log\left (\frac{4\pi \cdot d}{\lambda} \right)
\end{equation}

The wave length can also be described by a relationship between the frequency and the velocity of light. This relationship is described by the equation \ref{eq:vel_freq_wavelen1}.

\begin{equation}\label{eq:vel_freq_wavelen1}
\lambda = \frac{c}{f}
\end{equation}


Explanation of the parameters.
\begin{itemize}
\item d - distance from transmitter to receiver [m]
\item $\lambda$ - wavelength of the signal [m]
\item c - speed of light constant $3\cdot 10^8$ [m/s] 
\item f - frequency of signal [Hz]
\end{itemize}

Considering our problem we will look into the worst case scenario, which would be maximum distance between the base station and the UAV. 
\begin{equation*}
Assume \begin{cases}
d_{max} = \sqrt{x^2+y^2}\\
\text{f} = f\text{GHz (Should be permitted by law})\\
\end{cases}
\end{equation*}

Computing our signal wavelength:
\begin{equation}\label{eq:vel_freq_wavelen2}
	\lambda = \frac{c}{f} 
	        = \frac{3\cdot 10^{8}}{f\cdot 10^{9}}
	        = \frac{3}{10\cdot f}m
\end{equation}

Computing the path loss:
\begin{align*}\label{eq:path_loses_calc}
	L = 20lg\left (\frac{4\pi d}{\lambda} \right) dB &= 20lg\left (\frac{4\pi \sqrt{x^2+y^2}}{\frac{3}{10f}} \right) dB \\ 
	&= 20lg\left (\frac{4\pi \sqrt{x^2+y^2}\cdot 10f}{ 3} \right) dB
\end{align*}

\noindent \textbf{As an observation, higher distance value between basestation and UAV will result in higher path loss.}